\documentclass[a4paper, 12pt]{article}
\usepackage{cmap}
\usepackage[12pt]{extsizes}			
\usepackage{mathtext} 				
\usepackage[T2A]{fontenc}			
\usepackage[utf8]{inputenc}			
\usepackage[english,russian]{babel}
\usepackage{setspace}
\singlespacing
\usepackage{amsmath,amsfonts,amssymb,amsthm,mathtools}
\usepackage{fancyhdr}
\usepackage{soulutf8}
\usepackage{euscript}
\usepackage{mathrsfs}
\usepackage{listings}
\pagestyle{fancy}
\usepackage{indentfirst}
\usepackage[top=10mm]{geometry}
\rhead{}
\lhead{}
\renewcommand{\headrulewidth}{0mm}
\usepackage{tocloft}
\renewcommand{\cftsecleader}{\cftdotfill{\cftdotsep}}
\usepackage[dvipsnames]{xcolor}

\lstdefinestyle{mystyle}{ 
	keywordstyle=\color{OliveGreen},
	numberstyle=\tiny\color{Gray},
	stringstyle=\color{BurntOrange},
	basicstyle=\ttfamily\footnotesize,
	breakatwhitespace=false,         
	breaklines=true,                 
	captionpos=b,                    
	keepspaces=true,                 
	numbers=left,                    
	numbersep=5pt,                  
	showspaces=false,                
	showstringspaces=false,
	showtabs=false,                  
	tabsize=2
}

\lstset{style=mystyle}

\begin{document}
\thispagestyle{empty}	
\begin{center}
	Московский авиационный институт
	
	(Национальный исследовательский университет)
	
	Факультет "Информационные технологии и прикладная математика"
	
\end{center}
\vspace{40ex}
\begin{center}
	\textbf{\large{Лабораторная работа №3 по курсу \textquotedblleft Объектно-ориентированное программирование\textquotedblright}}
\end{center}
\vspace{40ex}
\begin{flushright}
	\textit{Студент: } Живалев Е.А.
	
	\vspace{2ex}
	\textit{Группа: } М8О-206Б
	
	\vspace{2ex}
	\textit{Преподаватель: } Журавлев А.А.
	
	\vspace{2ex}
	\textit{Вариант: } 5
	
	\vspace{2ex}
	\textit{Оценка: } \underline{\quad\quad\quad\quad\quad\quad}
	
	 \vspace{2ex}
	\textit{Дата: } \underline{\quad\quad\quad\quad\quad\quad}
	
\end{flushright}

\begin{vfill}
	\begin{center}
		Москва
		
		2019
	\end{center}	
\end{vfill}
\newpage
\section{Исходный код}

Ссылка на github : https://github.com/QElderDelta/oop\_exercise\_03

\vspace{3ex}
\textbf{\large{Figures.hpp}}
\lstinputlisting[language=C++]{Figures.hpp}

\vspace{3ex}
\textbf{\large{Figures.cpp}}
\lstinputlisting[language=C++]{Figures.cpp}

\vspace{3ex}
\textbf{\large{main.cpp}}
\lstinputlisting[language=C++]{main.cpp}

\vspace{3ex}
\textbf{\large{test.cpp}}
\lstinputlisting[language=C++]{test.cpp}

\vspace{3ex}
\textbf{\large{CMakeLists.txt}}
\lstinputlisting{CMakeLists.txt}
\newpage
\section{Тестирование}
\vspace{3ex}

\textbf{test\_01.txt}:

Попробуем добавить в вектор фигуру с координатами (-5, 0), (-4, -1), (-3, -1), (-2, 0), которая очевидно не является ромбом, рассчитывая получить сообщение об ошибке. Затем добавим в вектор ромб с координатами (-5, 0), (-3, 1), (-1, 0), (-3, -1), площадь которого равна 4, а центр находится в точке (-3, 0), а также пятиугольник с координатами (-3.000, 0.000), (-2.000, 1.000), (-1.000, 1.000), (0.000, 0.000), (-1.000, -1.000), площадь которого равна 3.5 и шестиугольник с координатами
(-3.000, 0.000), (-2.000, 1.000), (-1.000, 1.000), (0.000, 0.000), (-1.000, -1.000), (-2.000, -1.000), (-1.500, -0.000) с площадью равной 4. Попробуем удалить из вектора элемент, находящийся на 3 позиции, надеясь получить сообщение об ошибке. Затем выведем все фигуры, а также найдем общую площадь фигур в массиве, которая должна быть равна 11.5, затем удалим шестиугольник и еще раз выведем все фигуры.

Результат:

1 - add figure to the vector

2 - delete figure from the vector

3 - call common functions for the whole vector

4 - get total area of figures in vector

0 - exit

1 - Rhombus, 2 - Pentagon, 3 - Hexagon

Entered coordinates are not forming Rhombus. Try entering new coordinates

1 - Rhombus, 2 - Pentagon, 3 - Hexagon

Enter index

Element with such index doesn't exist

1 - Rhombus, 2 - Pentagon, 3 - Hexagon

7.5

Rhombus: [-5.000, 0.000], [-3.000, 1.000], [-1.000, 0.000], [-3.000, -1.000]

[-3.000, 0.000]

4.000

Pentagon: [-3.000, 0.000], [-2.000, 1.000], [-1.000, 1.000], [0.000, 0.000], [-1.000, -1.000]

[-1.429, 0.095]

3.500

1 - Rhombus, 2 - Pentagon, 3 - Hexagon

Rhombus: [-5.000, 0.000], [-3.000, 1.000], [-1.000, 0.000], [-3.000, -1.000]

[-3.000, 0.000]

4.000

Pentagon: [-3.000, 0.000], [-2.000, 1.000], [-1.000, 1.000], [0.000, 0.000], [-1.000, -1.000]

[-1.429, 0.095]

3.500

Hexagon: [-3.000, 0.000], [-2.000, 1.000], [-1.000, 1.000], [0.000, 0.000], [-1.000, -1.000], 
[-2.000, -1.000]

[-1.500, -0.000]

4.000

11.500

Enter index

Rhombus: [-5.000, 0.000], [-3.000, 1.000], [-1.000, 0.000], [-3.000, -1.000]

[-3.000, 0.000]

4.000

Pentagon: [-3.000, 0.000], [-2.000, 1.000], [-1.000, 1.000], [0.000, 0.000], [-1.000, -1.000]

[-1.429, 0.095]

3.500





\vspace{3ex}

\textbf{test\_02.txt} 

Добавим в вектор ромб с координатами [4.000, 0.000], [8.000, 2.000], [12.000, 0.000], [8.000, -2.000], центром в точке [8, 0] и площадью равной 16, квадрат с координатами [4.000, 2.000], [8.000, 2.000], [8.000, -2.000], [4.000, -2.000] с центром в точке [6, 0] и площадью равной 16, пятиугольник с координатами [4.000, 0.000], [8.000, 2.000], [12.000, 0.000], [8.000, -2.000], [6.000, -2.000] и площадью равной 18. Затем выведем все фигуры и найдем общую площадь, которая должна быть равна 50. Добавим шестиугольник с координатами [4.000, 0.000], [8.000, 2.000], [10.000, 2.000], [12.000, 0.000], [8.000, -2.000], [6.000, -2.000] и площадью равной 20. Еще раз выведем все фигуры и найдем общую площадь, которая должна быть равна 70. Затем удалим пятиугольник и шестиугольник и еще раз выведем все фигуры.

Результат:

1 - add figure to the vector

2 - delete figure from the vector

3 - call common functions for the whole vector

4 - get total area of figures in vector

0 - exit

1 - Rhombus, 2 - Pentagon, 3 - Hexagon

1 - Rhombus, 2 - Pentagon, 3 - Hexagon

1 - Rhombus, 2 - Pentagon, 3 - Hexagon

Rhombus: [4.000, 0.000], [8.000, 2.000], [12.000, 0.000], [8.000, -2.000]

[8.000, 0.000]

16.000

Rhombus: [4.000, 2.000], [8.000, 2.000], [8.000, -2.000], [4.000, -2.000]

[6.000, 0.000]

16.000

Pentagon: [4.000, 0.000], [8.000, 2.000], [12.000, 0.000], [8.000, -2.000], [6.000, -2.000]

[7.778, -0.148]

18.000

50.000

1 - Rhombus, 2 - Pentagon, 3 - Hexagon

Rhombus: [4.000, 0.000], [8.000, 2.000], [12.000, 0.000], [8.000, -2.000]

[8.000, 0.000]

16.000

Rhombus: [4.000, 2.000], [8.000, 2.000], [8.000, -2.000], [4.000, -2.000]

[6.000, 0.000]

16.000

Pentagon: [4.000, 0.000], [8.000, 2.000], [12.000, 0.000], [8.000, -2.000], [6.000, -2.000]

[7.778, -0.148]

18.000

Hexagon: [4.000, 0.000], [8.000, 2.000], [10.000, 2.000], [12.000, 0.000], [8.000, -2.000], 

[6.000, -2.000]

[8.000, 0.000]

20.000

70.000

Enter index

Enter index

Rhombus: [4.000, 0.000], [8.000, 2.000], [12.000, 0.000], [8.000, -2.000]

[8.000, 0.000]

16.000

Rhombus: [4.000, 2.000], [8.000, 2.000], [8.000, -2.000], [4.000, -2.000]

[6.000, 0.000]

16.000

\newpage

\section{Объяснение результатов работы программы}

При вводе координат для создания ромба производится проверка этих координат, ведь они могут не образовывать ромб. Для этого реализована функция checkIfRhombus, которая вычисляет расстояния от одной точки до трёх остальных, а поскольку фигура является ромбом, то два из низ должны быть равны. Третье же значение функция возвращает ведь оно равно длине одной из диагоналей. Площадь ромба вычисляется как половина произведения диагоналей, центр - точка пересечения диагоналей. Методы вычисления площади и центра для пяти- и шестиугольника совпадают. Чтобы найти площадь необходимо перебрать все ребра и сложить площади трапеций, ограниченных этими ребрами. Чтобы найти центр необходимо разбить фигуры на треугольники(найти одну точку внутри фигуры), для каждого треугольника найти центроид и площадь и перемножить их, просуммировать полученные величины и разделить на общую площадь фигуры.   

\newpage
\section{Выводы}

В ходе выполнения лабораторной работы я познакомился с таким  понятием как runtime-полиморфизм. Также я познакомился с библиотекой для юнит-тестов из коллекций библиотек boost, которую уже могу сравнить с ранее использованным googletest. На мой взгляд, googletest предоставляет чуть больше возможностей для тестирования кода и имеет более приятную организацию тестов.
\end{document}